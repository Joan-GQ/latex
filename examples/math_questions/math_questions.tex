\documentclass{article}

\usepackage{jig}

\customuline{}

\begin{document}

\section*{Ejercicios}
\begin{enumerate}[label=\textsc{\roman*}.]
    \item Para cada una de las siguientes ecuaciones, determinar en cada caso, si es posible, el o los valores que debe tomar la variable $x$ para que sean verdaderas:
        \begin{enumerate}[label=\arabic*.]
            \item $2x + 1 = 5$                                              % x = 2
            \item $-5x - 2 = 3$                                             % x = -1
            \item $6x + 2 = x$                                              % x = -2/5
            \item $6(x+2)-5(x-4) = 2(x+6)$                                  % x = 20
            \item $-2 (x - 3(5x - 4)) = 7(x-4)$                             % x = -4/21
            \item $\frac{x}{6}+2 = 2$                                       % x = 0
            \item $\frac{x}{7}-1 = 8$                                       % x = 63
            \item $\frac{11x}{6} - \frac{5}{6} = 1$                         % x = 1
            \item $\frac{5x}{3} + \frac{x}{4} = -0.5$                       % x = -6/23
            \item $x (\frac{1}{2}+3) - 2(\frac{x+2}{5}) = 8x+6$             % x = -68/49
            \item $-5(\frac{x-2}{5}) + 8(\frac{x-8}{7}) = \frac{x+2}{5}$    % x = -132
            \item $x^2 = 4$                                                 % x = +-2
            \item $x^2 = -1$                                                % No hay solución
            \item $(x+1)(x-2) = 3$                                          % x \in \{ -2, 3 \}
            \item $x^2 + 2x = 5x + 10$                                      % x \in \{ -2, 5 \}
            \item $x(x+1) - 6x + \frac{2}{3} = x$                           % x \in \{ 3 +- 5/sqrt(3) \}
        \end{enumerate}
    
    \item {Sea $f(x)$ la función definida por:
        \begin{equation*}
            f(x) = 2x^2 - x - 6
        \end{equation*}
        \begin{enumerate}[label=\arabic*.]
        \item { Determinar los valores de:
            \begin{tasks}[style=enumerate, after-item-skip=4mm, label = \alph*)](6)
                \task $f(0)$                    % -6
                \task $f(3)$                    % 9
                \task $f(-1)$                   % -3
                \task $f(0.5)$                  % -6
                \task $f(-\frac{1}{2})$         % -5
                \task $f(\sqrt 2)$              % -2 - sqrt(2) ~ -3,4142
            \end{tasks}
            }
        \item ¿Para qué valores de $x$ se cumple que $f(x) = 0$? ¿Y para $f(x) = -5$?
        % Raíces => x \in \{ -3/2, 2 \}, raíces de f(x)+5 => x \in \{ -1/2, 1 \}
        
        \item Teniendo en cuenta lo calculado previamente, dar un gráfico de la función indicando sus raíces.
        \\\\ \ul{Consejo:} Para obtener un mejor gráfico, puede ser útil calcular algunos valores más de la función. Por ejemplo, $f(\frac{1}{4}) = -\frac{49}{8}$ nos da el punto \\($\frac{1}{4}$, $-\frac{49}{8}$), que es el vértice de la parábola que grafica $f(x)$.
        \end{enumerate}
        
    }
    
    \newpage
    \item Para cada una de las siguientes funciones, dar sus raíces, si es que existen.
        \begin{tasks}[style=enumerate, after-item-skip=4mm, label = {\alph*)}](2)
            \task $f_1(x) = x^2 - 2$                                % +- sqrt(2)
            \task $f_2(x) = x^2 - 4x + 2$                           % 2 +- sqrt(2)
            \task $f_3(x) = 3 x^2 - 30 x + 48$                      % {2,8}
            \task $f_4(x) = \frac{3}{2} (x^2 - 2 x - 15)$           % {-3, 5}
            \task $f_5(x) = \frac{3 x^2}{2} + x$                    % x = -2/3
            \task $f_6(x) = (x+5)(x-2) \cdot (-\frac{2}{5})$        % {-5, 2}
            \task $f_7(x) = x^2 + 1$                                % No hay solución
            \task $f_8(x) = -x^2 + 1$                               % +- 1
            \task $f_9(x) = -x^2 - x - 5$                           % No hay solución
            \task $f_{10}(x) = x(x+5)$                              % -5
        \end{tasks}
    
    \ul{Consejo:} Dada una función de la forma $ax^2 + bx + c$, definimos:
        \begin{equation*}
            \Delta = b^2 - 4ac
        \end{equation*}
    A este número lo llamamos \textit{discriminante}, y lo denotamos con la letra griega delta mayúscula ($\Delta$). Nos ayuda a determinar si existen raíces reales o no, tal que:
    \begin{itemize}[label=$\cdot$]
        \item Si $\Delta > 0$, entonces existen exactamente 2 raíces reales.
        \item Si $\Delta = 0$, entonces existe exactamente 1 raíz real.
        \item Si $\Delta < 0$, entonces no existen soluciones reales.
    \end{itemize}
    
    \item Sea $g(x)$ la función definida por
    \begin{equation*}
        g(x) = x^4 - \frac{5 x^3}{6} - \frac{31 x^2}{2} - \frac{74 x}{3} - 10
    \end{equation*}
    Sabiendo que 
    \begin{equation*}
    (x^2 - 3 x - 10) \cdot (x^2 + \frac{13}{6}x + 1) = x^4 - \frac{5 x^3}{6} - \frac{31 x^2}{2} - \frac{74 x}{3} - 10.
    \end{equation*}
    \\ ¿Cuáles son las raíces de $g(x)$?
    % x \in \{ -2, -3/2, -2/3, 5\}
\end{enumerate}

\newpage
\section*{Respuestas}
\begin{enumerate}[label=\textsc{\roman*}.]
    \item    \begin{enumerate}[label=\arabic*.]
             \item $x = 2$
             \item $x = -1$
             \item $x = -\frac{2}{5}$
             \item $x = 20$
             \item $x = -\frac{4}{21}$
             \item $x = 0$
             \item $x = 63$
             \item $x = 1$
             \item $x = -\frac{6}{23}$
             \item $x = -\frac{68}{49}$
             \item $x = -132$
             \item $x = \pm 2$
             \item No hay soluciones reales.
             \item $\left\{\begin{array}{lr} x_1 =  -2 \\ x_2 = 3 \end{array} \right.$
             \item $\left\{\begin{array}{lr} x_1 =  -2 \\ x_2 = 5 \end{array} \right.$
             \item $\left\{\begin{array}{lr} x_1 =  3 + 5\sqrt3 \\ x_2 = 3 - 5\sqrt3 \end{array} \right.$
             \end{enumerate}
    \item { $\;$
        \begin{enumerate}[label=\arabic*.]
            \item \begin{enumerate}[label=\alph*)]
                    \item $f(0) = -6$
                    \item $f(3) = 9$
                    \item $f(-1) = -3$
                    \item $f(0.5) = -6$
                    \item $f(-\frac{1}{2}) = -5$
                    \item $f(\sqrt 2) = -2 - \sqrt2$
                  \end{enumerate}
            \item Las raíces de $f(x)$ son $\left\{\begin{array}{lr} x_1 =  -\frac{3}{2} \\ x_2 = 2 \end{array} \right.$
            \\ Los valores para los cuales $f(x) = -5$ son $\left\{\begin{array}{lr} x_1 =  -\frac{1}{2} \\ x_2 = 1 \end{array} \right.$
        \end{enumerate}
        
        \newpage
        
        \item $\;$
        \begin{tasks}[style=enumerate, after-item-skip=4mm, label = {\alph*)}](2)
            \task  $\left\{\begin{array}{lr} x_1 =  -2 \\ x_2 = 2 \end{array} \right.$
            \task  $\left\{\begin{array}{lr} x_1 =  -\sqrt2 \\ x_2 = \sqrt2 \end{array} \right.$
            \task  $\left\{\begin{array}{lr} x_1 =  2 \\ x_2 = 8 \end{array} \right.$
            \task  $\left\{\begin{array}{lr} x_1 =  -3 \\ x_2 = 5 \end{array} \right.$
            \task  $x = -\frac{2}{3}$
            \task  $\left\{\begin{array}{lr} x_1 =  -5 \\ x_2 = 2 \end{array} \right.$
            \task  No hay raíces reales.
            \task  $\left\{\begin{array}{lr} x_1 =  -1 \\ x_2 = 1 \end{array} \right.$
            \task  No hay raíces reales.
            \task  $x = -5$
        \end{tasks}
    }
    
    \item Las raíces de $g(x)$ son $\left\{\begin{array}{lr} x_1 =  -2 \\\\ x_2 = -\frac{3}{2} \\\\ x_3 = -\frac{2}{3} \\\\ x_4 = 5 \end{array} \right.$
\end{enumerate}

\end{document}